% Präambel für Master Thesis
% Dokumentenklasse
\documentclass[a4paper, 
                11pt, 
                twoside, 
                onecolumn, 
                openright,
                final]{memoir}

%% Pakete
% Unterscheidung zwischen compiler LuaLaTeX vs pdfLaTeX
\usepackage{ifluatex}
\ifluatex 
    % lualatex
    \usepackage{polyglossia}
    \setmainlanguage{german}
    \usepackage{fontspec}
    %\defaultfontfeatures{Ligatures=TeX}
    %\usepackage[]{unicode-math} 
    %\unimathsetup{math-style=TeX}
    \setmainfont[Ligatures=TeX]{DejaVuSerif} 
    \setsansfont{DejaVuSans}
    \setmonofont{DejaVuSansMono}
\else 
    % pdflatex
    \usepackage[T1]{fontenc}
    \usepackage[utf8]{inputenc}
    % Schriftarten
    \usepackage{palatino}       % Serifen %times
    \usepackage[scaled]{helvet} % Serifenlos
    \usepackage{courier}        % Schreibmaschinenschrift
    %\usepackage{mathptmx}      % Matheschrift
\fi

%% Sprache
%\usepackage[ngerman, british]{babel}


%% Allgemeines Layout
\setbinding{1cm} % bindekorrektur
% \semiisopage[] 
\semiisopage[9]
% \setlrmargins{*}{1cm}{*}
\OnehalfSpacing % zeilenabstand
\usepackage[document]{ragged2e} % linksbündig 
\usepackage[babel,german=quotes]{csquotes} % Anführungszeichen
\usepackage{color}                    % farbe 

%% Layout Kopf- und Fußzeile
% plain
\makeevenhead{plain}{}{}{}
\makeevenfoot{plain}{}{\thepage}{}
\makeoddhead{plain}{}{}{}
\makeoddfoot{plain}{}{\thepage}{}
% headings
\makeevenhead{headings}{\leftmark}{}{}
\makeevenfoot{headings}{}{\thepage}{}
\makeoddhead{headings}{}{}{\rightmark}
\makeoddfoot{headings}{}{\thepage}{}
% sep line
%\makeheadrule{plain}{\textwidth}{.5pt}
\makefootrule{plain}{\textwidth}{.5pt}{0ex}
\makeheadrule{headings}{\textwidth}{.5pt}
\makefootrule{headings}{\textwidth}{.5pt}{0ex}

%% Querverweise
\usepackage[linktoc=all, hypertexnames=false, plainpages=false, pdfpagelabels]{hyperref}
\usepackage{memhfixc}

%% Inhaltsverzeichnis
%\usepackage[nottoc,numbib]{tocbibind}
\setcounter{tocdepth}{1}

%% Matheumgebungen
\usepackage[fleqn]{mathtools}

%% SI-Einheiten
\usepackage{siunitx}
\sisetup{locale = DE,
          list-final-separator = \text{~und~},
          list-pair-separator = \text{~und~},
          range-phrase = \text{~bis~}}
\DeclareSIUnit\year{yr}
%\usepackage[german]{translator}

%% Abbildungen
\usepackage{tikz}
\usepackage{graphicx}
\usepackage{float} % für Fließumgebungen; Platzierung H verschiebt nicht
\restylefloat{figure}
\ifluatex
    \newcommand{\includegraphicstikz}{\input}
\else
    \newcommand{\includegraphicstikz}{\includegraphics}
\fi

%% Captions von Tabellen und Abbildungen
\makeatletter
\renewcommand{\fnum@table}[1]{\small \textbf{\tablename~\thetable:} }
\renewcommand{\fnum@figure}[1]{\small \textbf{\figurename~\thefigure:} }
\makeatother

%% Tabellen
% \usepackage{tabular}
\usepackage{tabularx}
\newcommand{\minitab}[2][l]{\begin{tabular}{#1}#2\end{tabular}}
\usepackage{multirow}
\usepackage{bigstrut}


%% Zitation und Literaturverzeichnis
\usepackage[square, semicolon]{natbib}
\bibliographystyle{abbrvnat} % plainnat abbrvnat unsrtnat
\usepackage{url}


%% Blindtexte
\usepackage{blindtext, lipsum}


%
%% ## Ende der Präambel
