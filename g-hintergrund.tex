\chapter{Hintergrund} \label{ch:hintergrund}

\citet{francis-2012} \\
Nachdem \citet{screen-2010} und \citet{serreze-2009} herausgearbeitet haben, dass sich die Arktis etwa doppelt so schnell erwärmt wie die gesamte nördliche Hemisphäre, beschäftigten sich \citet{francis-2012} mit der Frage, wie die großräumige atmosphärische Zirkulation von der im Winterhalbjahr vom Ozean abgegebenen Wärme, die durch die Aufnahme solarer Strahlung von größer werdende Freiwasserflächen zunimmt, beeinflusst wird. Einzelne Extremwettereignisse haben zwar typischerweise einen dynamischen Ursprung, resultieren jedoch aus anhaltenden Zirkulationsmustern, die häufig mit blockierenden Wellen mit hohen Amplituden verbunden werden. Als Beispiele hierfür werden die Hitzewellen 2010 in Europa und Russland, die Überschwemmungen am Mississippi 1993 und die Kälteereignisse in Florida im Winter 2010/11 genannt. 
Die Analyse unterstützt zwei Hypothesen, nach denen die arktische Amplifikation für anhaltende Zirkulationsmustern führen kann, die wiederum zu Extremereignissen führen können. Ein Effekt ist ein schwächerer polwärts gerichteter Gradient der $1000-500\,$hPa-Schichtdicke, was die Rossby-Welle auf eine höheren Amplitudenbahn drängt und so zu langsameren Zirkulationssystemen führt. Dies erhöht die Wahrscheinlichkeit für Extremwetterereignisse, die von persistenten Wetterlagen verursacht werden. Der zweite Effekt ist eine nordwärts gerichtete Verlagerung der Rücken der Wellen in $500\,$hPa Höhe, was zu einer höheren Amplitude der Strömung führt und die Wahrscheinlichkeit von Blocking-Situationen weiter erhöht. Die Studie schließt damit, dass durch das weitere Schwinden der arktischen Meereisbedeckung davon auszugehen ist, dass die großräumige Zirkulation zunehmend durch die arktische Amplifikation beeinflusst wird. 

\citet{barnes-2013-b}\\
In \citet{barnes-2013-b} wird untersucht, wie der Jetstream in den mittleren Breiten auf eine Zunahme der CO$_2$-Konzentration in der Atmosphäre reagiert. In allen untersuchten Regionen (Nordatlantik, Nordpazifik und südliche Hemisphäre) zeigen die Analysen eine Verschiebung des Jetstreams Richtung Pol um etwa $1-2^{\circ}$  zum Ende des 21. Jahrhunderts. Darüber hinaus verändert sich die Variabilität. So wird für den Nordatlantik eine eher pulsierender und weniger taumelnder Jet prognostiziert, während für den Nordpazifik der umgekehrte Fall gilt. Da einige Modelle starke Veränderungen in der Jetposition projizieren, ist es möglich, dass sich das Variabilitätsmuster in der Zukunft deutlich von bekannten Verhalten des Jetstreams unterscheidet. Die Jetstreamvaribilität ist stark verbunden mit den Zugbahnen von Tiefdruckgebieten, dem regionalem Wetter sowie mit Blockings in Hochdruckgebieten. Durch diese vielfältigen Verbindungen zu physikalischen Prozessen in der Troposphäre und der Erdoberfläche werden Veränderungen in den dominanten Variabilitätsformen der Jetstreams wichtige globale Auswirkungen haben.

\citet{petoukhov-2013} \\
Im Nachgang des außergewöhnlichen Sommers in 2003 in Europa schlugen \citet{schär-2004} vor, dass der beobachtete Klimatrend die Wahrscheinlichkeitsverteilung der Sommertemperaturen in Richtung wärmerer Werte verschiebt und diese Verteilung verbreitert, sodass Extremwerte wahrscheinlicher werden.  Jedoch erklären auch eine veränderte Wahrscheinlichkeitsverteilung die regionalen Sommerextrema der vergangenen Jahre nicht vollständig, sodass angenommen wird, dass kein rein stochastischer Extremwirkungsmechanismus am Werk ist \citep{}. Diese Hitzewellen in Russland in 2010 sowie in den Vereinigten Staaten in 2011, deren Muster sich räumlich über die gesamte Hemisphäre und zeitlich über den gesamten Sommer erstreckten, legen nahe, dass es sich nicht um typische Blocking-Lagen mit einer charakteristischen Zeit von fünf bis sieben Tagen handelt. Eine quasistationäre freie synoptisch-skalige Welle blockiert, sobald die beiden so genannte Wendepunkte der mittleren Breiten  für die Welle auftreten, sodass eine starke dynamische Reaktion auf die klimatologischen diabatischen und orographischen Antriebe begünstigt wird. Die vorliegenden Daten und Ergebnisse deuten auf eine Veränderung der atmosphärischen Bedingungen hin, so dass die betrachtete quasiresonante Wellenverstärkung häufiger auftreten könnte.

\citet{screen-2013}\\
Ziel von \citet{screen-2013} es, die atmosphärischen Veränderungen besser zu verstehen, die möglicherweise als Reaktion auf die beobachteten Rückgänge des arktischen Meereises in den letzten drei Jahrzehnten eingetreten sind. Die Ergebnisse aus Simulationen mit zwei unabhängigen allgemeinen generellen Zirkulationsmodellen, deren einziger Antrieb die Variationen der Beobachtungen des arktischen Meereises zwischen 1979 und 2009 waren, deuten darauf hin, dass atmosphärische Auswirkungen auf den Verlust des Meereises am stärksten innerhalt der maritimen und Küstenregionen der Arktis auftreten. Die Modelle legen nahe, dass die Abnahme des arktischen Meereises den Energietransfer vom Ozean in die Atmosphäre, die verstärkte Erwärmung und die zunehmende Feuchte der unteren Troposphäre, die Stärke der Inversion sowie die erhöhte Schichtdicke der unteren Troposphäre begünstigen. Die Modelle zeigen kaum Anzeichen für eine durch das Meereis hervorgerufene Temperaturänderung oberhalb der stabilen Grenzschicht, obwohl Beobachtungen und Analysen bereits auf eine Temperaturerhöhung hindeuten. Die beobachtete arktische Erwärmung in der Höhe wird wahrscheinlich durch Änderungen der Oberflächentemperatur des Meeres und die resultierende Zunahme des polumschlagenden Wärmetransports in die Arktis verursacht \citet{screen-2012}. Die Modellexperimente ermöglichen die Isolierung von durch Meereis verursachten atmosphärischen Veränderungen, die sich in gekoppelten Modellversuchen und in der Natur nur schwer entschlüsseln lassen. Die Analyse verschiedener Kälteperioden der vergangenen Jahre hat gezeigt, dass die Modelle keien robuste oder weit verbreitete Abkühlung oder erhöhte Schneefälle als Antwort auf die arktische Amplifikation in den letzten drei Jahrzehnten simulieren. Dies schließt zwar nicht aus, dass in den kalten Wintern 2009/10 und 2010/11 die Meereisbedingungen eine treibende Rolle spielten, aber es deutet darauf hin, dass die vorgeschlagenen Verbindungen zwischen multidekadalen Änderungen des Meereises und der borealen Winterkühlung möglicherweise verfrüht sind. 

\citet{barnes-2014}\\
\citet{liu-2012} \citet{francis-2012} \citet{tang-2013} zeigten, dass die Erwärmung der Arktis und die Abnahme von polarem Meereis zu einer Zunahme von Blockade-Situationen auf der nördlichen Hemisphäre geführt haben. Ob die Häufigkeit solcher Blockings in den letzten Jahrzehnten robuste Trends aufweisen, wird mit drei verschiedenen Methoden bei vier verschiedenen Reanalysen untersucht. Eine deutliche Zunahme von Blockings ist zu keiner Jahreszeit für den Blocking Index festzustellen, obwohl für einzelne Regionen und Jahreszeiten robuste Unterschiede zu verzeichnen sind. Die Blockingfrequenzen im September führen für Jahre mit einer gößeren Ausdehnung der Eisfläche verglichen mit solchen mit einer geringeren Ausdehnung zu gegensätzlichen Ergebnissen. Während es für die Sommermonate positive Unterschiede über dem Nordatlantik gibt, sind jene über dem Nordpazifik negativ. Es wird ausdrücklich  gewarnt, dass diese zusammengesetzten Unterschiede durch verschiedene dynamische Mechanismen erklärt werden können und nicht nur als Beleg für die Reaktion der Blockade auf den Verlust des Meereises angesehen werden sollten. Diese Schlussfolgerungen stützen die Schlussfolgerungen von\citet{barnes-2013}, dass der Zusammenhang zwischen der jüngsten arktischen Erwärmung und einer verstärkten Blockade der nördlichen Hemisphäre derzeit nicht durch Beobachtungen gestützt wird. Während das arktische Meereis in den letzten Jahren noch nie dagewesene Verluste erlitt, scheinen Blockings in diesen Jahren keine Ausnahme zu sein. Die große Variabilität der Blocking-Ereignisse, sowohl auf interannueller als auch auf dekadischer Zeitskala, unterstreicht die Schwierigkeit, eine potenziell erzwungene Reaktion von einer natürlichen Variabilität zu trennen.

\citet{francis-2015}\\
Nach \citet{francis-2015} verschärft sich der Klimawandel in der Arktis weiter, das schwinden des arktischen Meereises hält an \citep{}, die Masse des grönländischen Inlandeises nimmt ab \citep{}, die Schneebedeckung der nördlichen Hemisphäre während des Frühsommers geht zurück und die rasche Erwärmung der Arktis, die arktische Amplifikation, hält weiter an. Der überproportionale Temperaturanstieg beeinflusst die großräumige Zirkulation möglicherweise mit weitreichenden Auswirkungen. Ein wichtiger Treiber des polaren Jetstreams ist der meridionale Temperaturgradient, sodass bei fortschreitender Erwärmung eine Abschwächung der Zonalwinde zu erwarten ist. Weiter wurde angenommen, dass geringere Windgeschwindigkeiten den Jetstream särker mäandrieren lassen, was zu einer langsameren Propagation der Welle führen würde \citep{}. Der NOAA-Klima-Extrem-Index hat in den Vereinigten Staaten im Vergleich mit den Jahren vor der arktischen Amplifikation um etwa ein Drittel zugenommen. Noch ist unklar, ob die Erwärmung der Arktis hierfür eine Ursache ist. \citet{francis-2015} weisen nach, dass in Regionen und Jahreszeiten, in denen sich die polwärts gerichteten Gradienten als Reaktion auf die Arktis abgeschwächt haben, der Fluss der oberen Troposhphäre meridionaler oder welliger geworden ist. Darüber hinaus hat in den letzten Jahren die Häufigkeit von Tagen mit  Jetstreams mit hoher Amplitude zugenommen. Diese Muster, die sich durch hohe Amplituden auszeichnen, sind dafür bekannt, dass sie anhaltende Zirkulationsmuster erzeugen, die zu extremen Wetterereignissen führen können \citep{} \citep{}. Als Beispiele werden die kalten, schneereichen Winter im Osten der Vereinigten Staaten in den Wintern 2009/10, 2010/11 und 2013/14, die Rekordschneefälle in Japan und Südost-Alaska im Winter 2011/12 sowie die Überschwemmungen im Mittleren Osten im Winter 2012/2013 angeführt. Auf der Grundlage dieser Ergebnisse kommen \citet{francis-2015} zu dem Schluss, dass ein Fortschreiten der arktischen Amplifikation zu allen Jahreszeiten infolge des ungebremsten Anstiegs der Treibhausgasemissionen zu einem zunehmend welligen Charakter der Jetstreams der hohen Troposphäre und damit zu einer Zunahme extremer Wetterereignisse beitragen wird.

\citet{dicapua-2016} \\
\citet{dicapua-2016} haben sich damit beschäftigt, dass hochamplitudenreiche Rossby-Wellen in der Strömung mittlerer Breitengrade statistisch mit Wetterextremen an der Oberfläche verknüpft sind \citep{} \citep{}. Es wurden mehrere Mechanismen vorgeschlagen, wie  die Wirkungsweise der Rossby-Wellen aussehen könnte. Klassische Studien haben gezeigt, dass anomale Temperaturen der Meeresoberfläche im tropischen Pazifik quasistationäre Wellenzüge erzeugen können, die den Jetstream in den mittleren Breiten stören und damit verändern können \citep{} \citep{}. Jüngste Arbeiten von \citet{trenberth-xxxx} zeigen, dass einige der beobachteten  Zirkulationsanomalien in den letzten Jahren möglicherweise aus einer negativen Phase der Pacific Decadal Oscillation und den damit verbundenen tropischen Niederschlagsanomalien hervorgegangen sind, wobei die stärksten Veränderungen im Winter zu verzeichnen sind \citep{}. Um Indizien für eine Zunahme der Mäandrierung der Zirkulation in den mittleren Breiten zu finden, wird von \citet{dicapua-2015} ein neuer Mäander-Index eingeführt, der die Welligkeit der mittleren Troposphäre misst. Darüber hinaus repräsentiert der Index mögliche Positionsveränderungen der Wellenaktivität durch eine Suche nach dem Maximum der Welligkeit zu einem bestimmten Zeitpunkt. Der Mäanderindex erfasst die Gesamtwelligkeit in der Atmosphäre, die auch bei mäßiger Nord-Süd-Ausdehnung (d. h. wenn hohe Wellenzahlen dominieren) ausgeprägt sein kann. Unabhängig von den zugrunde liegenden Faktoren könnte die Tendenz zu quasi-stationären Mäandern während des amerikanischen Sommerhalbjahres zu einer höheren Persistenz der Witterung und längeren Hitze- und Dürreperioden beitragen. Ziel von \citet{dicapua-2016} ist es, die maximale Welligkeit in der Atmosphäre der nördlichen Hemisphären über einen Index zu analysieren, der nicht durch Bewegungen der Welle in vertikaler oder meridionaler Richtung beeinflusst wird. Es werden einige robuste und signifikante Veränderungen mit diesem Index erkannt, die für Extreme im Bereich der mittleren Breiten relevant und die mit stark mäandernden Strömungsmustern verbunden sind.


...
\citep{barnes-2013-a}
\citep{cohen-2014}
\citep{chemke-2014}
\citep{coumou-2015}
\citep{kornhuber-2016}








%\citet{kornhuber-2016}
%Die Beobachtungen zeigen einen Anstieg der Häufigkeit und Schwere von Hitzeextremen und Starkregenereignissen in den mittleren Breitengraden NH (Coumou und Robinson 2013; Min et al. 2011; Westra et al. 2013; IPCC 2012; Hansen et al. 2012). Diese Ereignisse haben oft massive humanitäre und wirtschaftliche Auswirkungen (Coumou und Rahmstorf 2012). Wie aus der Thermodynamik ersichtlich, wird die Häufigkeit extremer Temperaturen und Starkregenereignisse in einem wärmenden Klima zunehmen (Rahmstorf und Coumou 2011; Christidis et al. 2014; Min et al. 2011). Mit der Verschiebung der mittleren Temperatur in Richtung höherer Werte werden rekordverdächtige oder überhöhte Temperaturen häufiger auftreten (Coumou und Robinson 2013; Coumou et al. 2013). Außerdem kann wärmere Luft mehr Wasserdampf aufnehmen, was wiederum zu einer Zunahme von Starkregenereignissen auf globaler Ebene führt (Westra et al. 2013; Lehmann et al. 2015; Fischer und Knutti 2015). Dennoch gehen einige der jüngsten Wetterextreme über das hinaus, was bei einer einfachen Verschiebung der Verteilung zu erwarten wäre. Luterbacher et al. (2004) schätzten für ein Extremereignis eine Wiederkehrperiode als europäische Hitzewelle 2003 von etwa 100 Jahren. Dieses Ereignis wurde jedoch bereits im selben Jahrzehnt von der Hitzewelle 2010 in Moskau übertroffen (Dole et al. 2011; Christidis et al. 2014), die kürzlich in einer globalen Analyse als die intensivste Hitzewelle quantifiziert wurde (Russo et al. 2014). Bei den genannten Extremereignissen war die mittlere Breitenzirkulation durch anomal persistente Mäandermuster des umgelagerten Jetstreams gekennzeichnet. Hemisphärenweite, hochamplitudenreiche Wellen blieben mehrere Wochen lang in derselben Längslage und verursachten dadurch kontinentale Extremwetterlagen (Petoukhov et al. 2013; Coumou et al. 2014; Ogi et al. 2005; Tachibana et al. 2010; Lau und Kim 2012; Black et al. 2004). Diese Fälle von anomalem Sommerwetter lassen darauf schließen, dass neben thermo-dynamischen Effekten auch dynamische Prozesse wie die Veränderung der Charakteristika der außertropischen Strahlströme und atmosphärischen Wellen eine Rolle für die beobachtete Zunahme von Häufigkeit und Schwere der jüngsten Sommerwetterextreme der nördlichen Hemisphäre spielen (Tachibana et al. 2010; Schubert et al. 2011; Petoukhov et al. 2013; Coumou et al. 2014; Ogi Hinweise auf dynamische Veränderungen der NH-Sommerzirkulation wurden in mehreren Studien nachgewiesen (Overland et al. 2012). Coumou et al. (2015) berichten von einer Schwächung verschiedener Schlüsselmerkmale wie zonaler Zonenmittelwert, kinetische Energie transienter synoptischer Wirbel und die Amplitude der sich schnell bewegenden Rossby-Wellen von 1979-2014. Horton et al. (2015) zeigen anhand selbstorganisierender Karten Clusteranalysen, dass im Sommer die Häufigkeit der antizyklonalen Zirkulation über Europa, Westasien und Ost-Nordamerika zugenommen hat und auch länger andauert. Sie analysieren jedoch 500 MB Geopotentialhöhen und es bleibt fraglich, ob die berichteten Veränderungen wirklich dynamische Effekte sind oder ob es sich um thermische Ausdehnungen der unteren Troposphäre oder eine Kombination aus beidem handelt. Dennoch könnten solche dynamischen Veränderungen zu verlängerten Hitzewellen in diesen Regionen beigetragen haben, aber ein physikalischer Mechanismus für die verlängerte Dauer dieser regionalen antizyklonalen Strömungsmuster ist nicht vorgesehen. Quasi-resonante Wellenverstärkung (QRA) wurde als ein Mechanismus vorgeschlagen, der zu gleichzeitigen Blockierungsereignissen innerhalb der mittleren Breiten führen könnte, indem er langsame Wellen einfängt und verstärkt (Petoukhov et al. 2013). Dieser interne atmosphärisch-dynamische Mechanismus ruft die Resonanz zwischen den verschiedenen Wellentypen hervor, was zu ihrer Verstärkung führt. Generell ist die großflächige azonale atmosphärische Zirkulation in mittlerer Breite durch zwei Arten von Wellen charakterisiert:  
%Synoptisch skalierte Rossby-Wellen: Diese Wellen haben eine Wellenzahl von 6 und mehr, relativ große Phasengeschwindigkeiten (Ostausbreitung in der Größenordnung von 6-12 m/s) und nur eine kleine quasistationäre Komponente (Schneider et al. 2014). Einmal entstanden, bedürfen sie keiner Forcierung mehr, um aufrechterhalten zu werden und werden daher als freie Wellen bezeichnet. Mathematisch können diese Wellen durch die azonale Stromfunktionsgleichung mit Null-Rechts-Gleichung (d. h. kein Forcen) beschrieben werden. 
%Forcierte Wellen: Diese Wellen sind das Ergebnis quasi-stationärer diabatischer oder orographischer forcings. Da sich die großräumigen Triebmuster auf wesentlich längeren Zeitskalen von mehreren Wochen verändern, können diese Wellen als quasistationär mit typischen Wellenzahlen kleiner als 6 angesehen werden. Normalerweise ist die erzwungene quasistationäre Komponente der Wellenzahlen 6-10 schwach und ihre Energie wird effektiv in Richtung der Pole und des Äquators verteilt. In den letzten Sommern mit schweren Wetterereignissen (u. a. der Hitzewelle in Europa 2010 und der Hitzewelle in Moskau 2010) wurden jedoch zonenförmig verlängerte Züge mit quasistationären Wellen mit hoher Amplitude und einer Wellenzahl von 6-8 beobachtet (Petoukhov et al. 2013). Wenn die Breitenverteilung der zonalen mittleren zonalen Windverteilung in den mittleren Breiten spezifische Eigenschaften aufweist (detaillierte Beschreibung in Abschnitt 2), kann die synoptische Skala quasistationäre freie Wellen in den mittleren Breiten fast vollständig eingefangen werden. Dies verhindert die schnelle Dissipation und meridionale Dispersion ihrer Energie und führt so zu ihrer Einschränkung in den mittleren Breiten. Dies tritt auf, wenn für diese Wellen zwei Reflexionspunkte (oder Wendepunkte) mit mittlerer Breite in unterschiedlichen Breiten auftreten. Diese Konstellation von Wendepunkten wird als Wellenleiter bezeichnet (Hoskins and Karoly 1981; Ambrizzi et al. 1995). Wenn zusätzlich die eingeschlossenen freien Wellen ähnliche Längenskalen wie die quasistationären Forcings haben, kann es durch Resonanzen zu einer deutlichen Vergrößerung der langsam laufenden Zwangswellen kommen. Dies ist die Essenz der QRA und kann zu hemisphärisch-weiten, persistenten Blockierungsmustern in den mittleren Breiten mit ungewöhnlich hohen Amplituden der zonalen Wellenzahlen 6,7 oder 8 führen, wie sie bei mehreren jüngsten extremen Wetterereignissen beobachtet wurden (Petoukhov et al. 2013; Coumou et al. 2014). Die QRA-Hypothese baut auf der klassischen Arbeit von Hoskins und Karoly (1981) auf, verfolgt aber in ihrer aktuellen Version einen zonal-mean-Ansatz in Bezug auf Wellenleiter. Wir tun dies, um die jüngsten anomalen Zirkulationsmuster zu erklären, die sich umweltglobal und stark zonal orientiert waren. Die Grenzen dieses Ansatzes werden in der Diskussion weiter erörtert (Abschnitt 4). Da die NH-Zirkulation starke saisonale Unterschiede aufweist (Bogenschütze und Caldeira 2008), mit schwächeren zonalen Fluss- und Wellenphasengeschwindigkeiten im Sommer, konzentrieren wir uns auf die Sommerzirkulation, nachdem frühere Studien die QRA-Ereignisse (Petoukhov et al. 2013; Coumou et al. 2014) analysiert hatten, die monatlich mittlere Daten verwendeten und nur auf Juli und August beschränkt waren. In dieser Studie stellen wir ein neuartiges Detektionsschema vor, das durch die Verwendung der von Petoukhov et al. (2013) abgeleiteten Bedingungen als diagnostisches Werkzeug zum Scannen von Reanalysedaten und Klimamodelldaten für QRA-Ereignisse unter Verwendung von 15-Tage-Mittelwertdaten verwendet werden kann.

%Indem wir die von Petoukhov et al. (2013) postulierten Bedingungen für die Verstärkung quasistationärer Wellen (QRA) in ein Detektionsschema einfließen ließen, etablierten wir eine objektive Methode, um große Datenmengen für QRA-Ereignisse zu scannen. Bei der Anwendung auf Reanalysedaten (1979-2015) zeigt das Schema, dass bei etwa einem Drittel aller Ereignisse mit hoher Amplitude (>1,5?) für die Wellen 6,7,8 die QRA-Bedingungen erfüllt sind. Die spektrale Analyse von QRA-Ereignissen zeigt einen erhöhten Anteil von langsam laufenden Wellen mit hoher Amplitude im Vergleich zur Sommerklimatologie. Für Welle 6 und Welle 7 ist diese Verschiebung der Wellenkennlinie statistisch signifikant, wobei Welle 8 die gleiche Tendenz ohne statistische Signifikanz zeigt. Diese Ergebnisse belegen die Gültigkeit der Hypothese und der zu Grunde liegenden Annahmen. Bei der Analyse der QRA-Ereignisse mit der höchsten Amplitude werden wir feststellen, dass die QRA-Erkennung der maximalen Wellenamplitude um etwa eine Woche vorausgeht. Viele der detektierten Ereignisse waren mit einem Doppelstrahl verbunden, der im zonal gemittelten zonalen Zonenwind sichtbar war. Anomale halbkugelförmige Witterungsbedingungen traten während der Mehrzahl der hier vorgestellten QRA-Perioden auf, und wir liefern statistische Daten, dass QRA-Ereignisse zu Extremwetterereignissen an der Oberfläche führen. Unsere Ergebnisse deuten darauf hin, dass diese Ereignisse mit ortsfesten Wellen mit hohen Amplituden über die mittleren Breiten verbunden waren. QRA-Perioden stimmen mit der Wellenverstärkung und damit der Ausrichtung der meridionalen Windgeschwindigkeiten (d. h. Verringerung der Phasengeschwindigkeit) überein. Eine Gruppe von verlängerten Resonanzen findet sich nach 2000 mit 6 von 9 Ereignissen, die nach 2000 eintreten (obwohl ihre Bedeutung unklar ist), viele davon fallen mit den Wetterextremen der mittleren Breitengrade NH zusammen. Dieser Cluster war in erster Linie auf lang anhaltende Wellenresonanz-Ereignisse zurückzuführen. Extreme Oberflächenwetterereignisse ereigneten sich vor allem in den Regionen in der Nähe starker Meridionalwinde. Verstärkte Wellen (oder genauer: Resonanzen) allein führen aber nicht unbedingt zu Extremen, sondern schaffen günstige Bedingungen für das Auftreten von Extremen (Screen und Simmonds 2014). Durch die Erzeugung von hemisphärisch weiträumigen quasistationären Wellen mit hoher Amplitude setzt QRA die Bühne, auf der extremes Wetter wahrscheinlicher ist. Faktoren wie Wellenphase (Einstellung von Graten und Mulden), Bodenfeuchte (Miralles et al. 2014) und synoptische Wetterbedingungen können verlängerte Wetterbedingungen in Extreme verwandeln. Um diesen Zusammenhang weiter zu untersuchen, werden in der Zukunft weitere Untersuchungen des Zusammenspiels von Doppelstrahlstrukturen, QRA und extremen Wetterereignissen anstehen. Darüber hinaus ermöglicht uns dieses operative QRA-Erkennungsschema, die Repräsentation des QRA-Mechanismus in Klimamodellen zu untersuchen. Durch die Anwendung des Schemas auf das Klimamodell können statistische Analysen von QRA-Ereignissen im Rahmen zukünftiger Projektionen verschiedener Klimaszenarien durchgeführt werden. Ob CMIP5-Modelle Resonanzereignisse exakt reproduzieren können, ist von besonderem Interesse, wenn man bedenkt, dass die Modelle Verzerrungen im Zusammenhang mit der Sommer-Rossby-Wellenaktivität gekannt haben (Schubert et al. 2011). Das Detektionsschema dient auch als erster Schritt zu einem besseren Verständnis der zugrundeliegenden Treiber. Um den QRA-Mechanismus und seine Rolle bei der Generierung von Wetterextremen weiter zu erforschen, wird das Detektionsschema auf die südliche Hemisphäre angewandt. Australien erlebte vor kurzem eine Reihe von extremen Wetterereignissen, einschließlich extremer Niederschläge und Dürreereignisse, die einer Erklärung bedürfen, und QRA könnte ein potenzieller Kandidat sein.

%\citet{barnes-2013-a}
%\citet{barnes-2013-a} quantifizieren beobachtete Trends im meridionalen Ausdehnung von Wellen über dem Nordatlantik/Nordamerika mit Hilfe von zwei verschiedenen Maßen und drei Reanalysen. Die Maße stimmen nicht überein, ob ein signifikanter Trend in der Amplitude der Welle beobachtet wurde. Diese Uneinigkeit entstammt den Autoren zufolge aus den unterschiedlichen Definitionen 
% und wir erklären diese Meinungsverschiedenheit als sich aufgrund der Methodik der Definition der Welle entweder auf täglicher oder saisonaler Zeitskala. Darüber hinaus zeigen wir, dass, wenn beide Metriken sich auf eine schmale Reihe von Isoplethen konzentrieren, um die Höhenrücken und Tröge einer vorbeiziehenden Welle zu verfolgen, sie fälschlicherweise eine Verschiebung des geopotentialen Höhenfeldes als eine Änderung der Wellenausdehnung interpretieren. Wenn diese Verschiebung berücksichtigt wird, ist kein signifikanter Trend festzustellen. Wir untersuchen weiter, ob sich die Massenwellen in den letzten Jahrzehnten verlangsamt haben und finden keine signifikanten Trends außer in den Herbstmonaten, obwohl die Bedeutung dieses Trends für das Diagnosefeld und die spezifische Mittelwertbildung sensibel ist. Darüber hinaus ist zu keiner Jahreszeit ein signifikanter Anstieg des Auftretens von Blockaden festzustellen. Wir kommen zu dem Schluss, dass der Mechanismus, der durch frühere Studien[z. B. Francis und Vavrus, 2012; Liu et al., 2012] vorgetragen wurde, dass eine verstärkte polare Erwärmung zu einem vermehrten Auftreten von sich langsam bewegenden Wettermustern und Blockaden geführt hat, durch die Beobachtungen nicht unterstützt erscheint. Eine aktuelle Studie von Screen and Simmonds[2013] zeigt auch, dass die von FV12 vorgeschlagene Entwicklung der planetarischen Wellen ein Artefakt der Methodologie sein könnte. Sie zeigen, dass ein alternativer Wert, der unempfindlich gegenüber einer Verschiebung von Z500 ist, keine signifikant positiven Trends in der Wellenamplitude hervorruft. Die hier vorgestellten Ergebnisse deuten weiter darauf hin, dass die von FV12 berichtete Wellenlängung zumindest teilweise ein Artefakt der Polverschiebung der Isoplethen mit polarer Erwärmung ist. Die Arktis verändert sich rasant, und diese Veränderungen werden voraussichtlich tiefgreifende Auswirkungen auf die nördliche Hemisphäre haben. Diese Studie zeigt jedoch, dass die Beziehung zwischen der arktischen Verstärkung und dem Wetter in der Mittelgebirgsmitte kompliziert ist. Zusätzliche Einflüsse aus anderen Breitengraden sowie interne Variabilität[Screen et al., 2013] spielen wahrscheinlich eine wichtige Rolle bei der Bestimmung der atmosphärischen Netto-Trends, und gezielte Modellierungsstudien sind notwendig, um die relative Bedeutung der Polarveränderungen auf das atlantische Wetter zu quantifizieren.










